% =============================================================================
% 附录:消融研究报告模板(重设计版)
% =============================================================================
% 本模板提供了一个完整的消融研究报告结构,包含所有必要部分。
% 只有需要填写的内容被框出(使用 \fbox),其余部分为普通文本。
% 每个框内都有填写说明。
% 您可以根据需要删除占位框,直接使用模板。
% =============================================================================

% 定义指令命令,用于突出显示填写说明
\newcommand{\instruction}[1]{\textbf{[#1]}}
\section*{附录:消融研究报告模板}

\begin{center}
    \Large \textbf{消融研究报告模板} \\
    \vspace{0.5cm}
    \normalsize \instruction{请根据您的实验内容填写以下各部分}
\end{center}

% -----------------------------------------------------------------------------
% 模板使用说明(移至开头)
% -----------------------------------------------------------------------------
\subsection*{模板使用说明}
\begin{itemize}
    \item 本模板提供了消融研究报告的标准结构,所有\instruction{占位框}内的内容均可直接替换。
    \item 建议使用 LaTeX 编辑器(如 Overleaf, TeXShop, VS Code + LaTeX Workshop)进行编辑。
    \item 如果您希望删除占位框,只需删除 \texttt{\textbackslash fbox} 和对应的 \texttt{\textbackslash begin\{minipage\}} ... \texttt{\textbackslash end\{minipage\}},保留内部内容即可。
    \item 每个部分上方的注释(以 \% 开头)提供了填写指导,撰写时请阅读。
    \item 图表请使用 \texttt{\textbackslash includegraphics} 插入,表格请使用 \texttt{tabular} 环境。
    \item 参考文献建议使用 BibTeX 管理,此处仅为示例。
\end{itemize}

\newpage
\textbf{\Large [消融研究报告模版]}
{\large \instruction{请在此处填写您的消融研究标题}}\\[0.5cm]
\vspace{0.3cm}
% -----------------------------------------------------------------------------
% 1. 标题与作者信息
% -----------------------------------------------------------------------------
\subsection*{1. 标题与作者信息}

% 注释:在此处填写报告标题、作者姓名、单位、邮箱等信息。
% 如果您需要更复杂的作者列表,可以使用 \author 命令。

\begin{center}
    \fbox{
        \begin{minipage}[t]{\textwidth}
            \textbf{作者:} \instruction{姓名1,姓名2}\\[0.2cm]
            \textbf{单位:} \instruction{单位名称}\\[0.2cm]
            \textbf{邮箱:} \instruction{email@example.com}\\[0.2cm]
            \textbf{日期:} \today
            \vspace{0.3cm}
        \end{minipage}
    }
\end{center}

% -----------------------------------------------------------------------------
% 2. 摘要
% -----------------------------------------------------------------------------
\subsection*{2. 摘要}

% 注释:摘要应简明扼要地概括研究背景、方法、主要结果和结论。
% 建议长度在150-250字之间。

\begin{center}
\fbox{
    \begin{minipage}[t]{\textwidth}
        \vspace{0.3cm}
        \textbf{摘要:}\\[0.2cm]
        \instruction{在此处填写摘要内容}。消融研究是通过系统地移除或修改模型的某个组件,观察性能变化,从而理解该组件贡献的实验方法。本研究针对\instruction{任务名称}任务,构建了基线模型\instruction{模型名称},并设计了\instruction{数字}个消融实验,分别考察了\instruction{组件1}、\instruction{组件2}、\instruction{组件3}等组件的影响。实验结果表明:\instruction{简要描述主要发现}。本研究为\instruction{领域}提供了设计指导,并验证了\instruction{某个观点}的重要性。
        \vspace{0.3cm}
    \end{minipage}
}
\end{center}

% -----------------------------------------------------------------------------
% 3. 关键词
% -----------------------------------------------------------------------------
\subsection*{3. 关键词}

% 注释:列出3-5个关键词,用逗号分隔。

\begin{center}
\fbox{
    \begin{minipage}[t]{\textwidth}
        \vspace{0.3cm}
        \textbf{关键词:} 消融研究,卷积神经网络,组件分析,模型设计,深度学习
        \vspace{0.3cm}
    \end{minipage}
}
\end{center}

% -----------------------------------------------------------------------------
% 4. 引言
% -----------------------------------------------------------------------------
\subsection*{4. 引言}

% 注释:引言部分应介绍研究背景、问题定义、相关工作、本研究的目标与贡献。

\begin{center}
\fbox{
    \begin{minipage}[t]{\textwidth}
        \vspace{0.3cm}
        深度学习模型通常由多个组件构成,例如卷积层、池化层、激活函数、归一化层、正则化技术等。理解每个组件对模型性能的贡献对于模型设计、优化和可解释性至关重要。消融研究(Ablation Study)是一种通过逐步移除或修改模型组件来评估其重要性的实验方法。

        本文针对\instruction{具体任务,如图像分类、目标检测等}任务,开展系统的消融研究。我们首先构建一个基线模型,然后设计一系列消融实验,分别考察\instruction{组件列表}等组件的影响。本研究的主要贡献包括:
        \begin{enumerate}
            \item 量化了各组件在\instruction{任务名称}任务中的重要性;
            \item 提出了针对\instruction{模型类型}的设计建议;
            \item 验证了\instruction{某个假设或观点};
            \item 提供了可复现的实验代码和详细的数据分析。
        \end{enumerate}

        本文结构如下:第5节介绍实验设计,包括基线模型和消融方案;第6节展示实验结果并进行定量分析;第7节讨论实验发现的实际意义;第8节总结全文并展望未来工作。
        \vspace{0.3cm}
    \end{minipage}
}
\end{center}

% -----------------------------------------------------------------------------
% 5. 实验设计
% -----------------------------------------------------------------------------
\subsection*{5. 实验设计}

% 注释:详细描述基线模型、消融变量、数据集、评估指标、训练设置等。
\textbf{5.1 基线模型}
\begin{center}
\fbox{
    \begin{minipage}[t]{0.9\textwidth}
        我们采用\instruction{模型名称}作为基线模型,其结构如表1所示。该模型包含\instruction{数字}个卷积层、\instruction{数字}个池化层、\instruction{数字}个全连接层,使用了\instruction{激活函数类型}、\instruction{归一化方法}和\instruction{正则化技术}。具体参数配置如下:
        \begin{itemize}
            \item 输入尺寸:\instruction{宽度×高度×通道数}
            \item 卷积核大小:3×3,步长1,填充1
            \item 池化:2×2最大池化,步长2
            \item 优化器:Adam,学习率0.001
            \item 损失函数:交叉熵损失
            \item 训练周期:50个epoch,批量大小64
        \end{itemize}
\end{minipage}
}
\end{center}
\vspace{0.3cm}

\textbf{5.2 消融方案}
\begin{center}
\fbox{
    \begin{minipage}[t]{0.9\textwidth}
        我们设计了以下消融实验,每次只改变一个变量,保持其他设置不变:
        \begin{enumerate}
            \item \textbf{实验A:} 移除卷积层(减少特征提取能力)
            \item \textbf{实验B:} 更换激活函数(ReLU $\rightarrow$ Sigmoid/Tanh)
            \item \textbf{实验C:} 移除批归一化层
            \item \textbf{实验D:} 移除Dropout正则化
            \item \textbf{实验E:} 改变卷积核大小(3×3 $\rightarrow$ 5×5)
            \item \textbf{实验F:} 更换池化类型(最大池化 $\rightarrow$ 平均池化)
        \end{enumerate}

    \end{minipage}
}
\end{center}
\vspace{0.3cm}

\textbf{5.3 数据集与评估指标}
\begin{center}
\fbox{
    \begin{minipage}[t]{0.9\textwidth}        
        实验使用\instruction{数据集名称}数据集,该数据集包含\instruction{数量}个训练样本和\instruction{数量}个测试样本,共\instruction{类别数}个类别。我们采用以下评估指标:
        \begin{itemize}
            \item \textbf{准确率(Accuracy)}:分类正确的样本比例
            \item \textbf{损失(Loss)}:交叉熵损失值
            \item \textbf{参数量(Parameters)}:模型总参数个数
            \item \textbf{计算量(FLOPs)}:前向传播的浮点运算次数
            \item \textbf{收敛速度}:达到特定准确率所需的epoch数
        \end{itemize}
        \vspace{0.3cm}
    \end{minipage}
}
\end{center}
\vspace{0.3cm}

% -----------------------------------------------------------------------------
% 6. 实验结果
% -----------------------------------------------------------------------------
\subsection*{6. 实验结果}

% 注释:以表格和图表形式展示定量结果,并对结果进行简要描述。

\begin{center}
\fbox{
    \begin{minipage}[t]{\textwidth}
        \vspace{0.3cm}
        \textbf{6.1 定量结果}\\[0.2cm]
        表1汇总了各消融实验的测试准确率、参数量、计算量和收敛速度。

        \begin{table}[H]
        \centering
        \begin{tabular}{lcccc}
        \toprule
        \textbf{实验} & \textbf{测试准确率} & \textbf{参数量} & \textbf{计算量(FLOPs)} & \textbf{收敛epoch} \\
        \midrule
        基线模型 & 78.3\% & 1.2M & 1.2G & 15 \\
        实验A(无卷积层X) & 65.7\% & 0.8M & 0.9G & 20 \\
        实验B(Sigmoid激活) & 62.5\% & 1.2M & 1.2G & 25 \\
        实验C(无BN) & 75.1\% & 1.2M & 1.2G & 18 \\
        实验D(无Dropout) & 76.8\% & 1.2M & 1.2G & 14 \\
        实验E(5×5卷积核) & 79.1\% & 1.8M & 2.1G & 12 \\
        实验F(平均池化) & 77.8\% & 1.2M & 1.2G & 16 \\
        \bottomrule
        \end{tabular}
        \caption{消融实验结果汇总}
        \end{table}

        \textbf{6.2 可视化分析}\\[0.2cm]
        图1展示了基线模型与消融模型的训练损失曲线,图2展示了验证准确率曲线。

        % 占位图
        \begin{center}
            \framebox[0.8\textwidth][c]{\parbox{0.7\textwidth}{\centering 图1:训练损失曲线\instruction{请替换为实际图像}}}
        \end{center}

        \begin{center}
            \framebox[0.8\textwidth][c]{\parbox{0.7\textwidth}{\centering 图2:验证准确率曲线\instruction{请替换为实际图像}}}
        \end{center}
        \vspace{0.3cm}
    \end{minipage}
}
\end{center}

% -----------------------------------------------------------------------------
% 7. 分析与讨论
% -----------------------------------------------------------------------------
\subsection*{7. 分析与讨论}

% 注释:对实验结果进行深入分析,解释每个组件的作用,讨论可能的原因,并与相关工作比较。

\begin{center}
\fbox{
    \begin{minipage}[t]{\textwidth}
        \vspace{0.3cm}
        \textbf{7.1 组件重要性分析}\\[0.2cm]
        根据表1的结果,我们可以对组件的重要性进行排序:
        \begin{enumerate}
            \item \textbf{卷积层}:移除后准确率下降12.6\%,说明卷积层是特征提取的核心。
            \item \textbf{激活函数}:将ReLU替换为Sigmoid导致准确率下降15.8\%,表明非线性激活的选择至关重要。
            \item \textbf{批归一化}:移除BN后准确率下降3.2\%,但收敛速度变慢,说明BN主要加速训练。
            \item \textbf{Dropout}:移除Dropout后准确率下降1.5\%,但过拟合风险增加。
            \item \textbf{卷积核大小}:增大卷积核带来0.8\%的提升,但计算量增加75\%。
            \item \textbf{池化类型}:最大池化与平均池化差异较小(0.5\%),说明池化类型对分类任务影响有限。
        \end{enumerate}

        \textbf{7.2 实际意义与设计建议}\\[0.2cm]
        基于以上分析,我们提出以下设计建议:
        \begin{itemize}
            \item 在资源受限的场景下,可以适当减少卷积层数,但至少保留2层。
            \item 激活函数应优先选择ReLU或其变体(Leaky ReLU, Swish)。
            \item 批归一化应成为标准配置,尤其当训练数据分布不稳定时。
            \item Dropout率建议设置在0.3–0.5之间,以平衡正则化与收敛速度。
            \item 卷积核大小推荐3×3,大卷积核可用多个小卷积核替代。
            \item 池化类型可根据任务选择:分类任务用最大池化,定位任务用平均池化。
        \end{itemize}

        \textbf{7.3 局限性}\\[0.2cm]
        本研究的局限性包括:
        \begin{itemize}
            \item 实验仅在一个数据集上进行,结论可能不具备普适性。
            \item 未考虑组件之间的交互效应(如BN与Dropout的交互)。
        \end{itemize}
        \vspace{0.3cm}
    \end{minipage}
}
\end{center}

% -----------------------------------------------------------------------------
% 8. 结论与未来工作
% -----------------------------------------------------------------------------
\subsection*{8. 结论与未来工作}

% 注释:总结全文,重申主要发现,并提出未来研究方向。

\begin{center}
\fbox{
    \begin{minipage}[t]{\textwidth}
        \vspace{0.3cm}
        \textbf{8.1 结论}\\[0.2cm]
        本文通过系统的消融研究,深入分析了CNN各组件在\instruction{任务名称}任务中的作用。主要结论如下:
        \begin{enumerate}
            \item 卷积层是CNN的核心组件,其数量与模型性能强相关。
            \item 激活函数的选择对模型性能影响显著,ReLU在大多数情况下表现最佳。
            \item 批归一化能显著加速训练,提高模型稳定性。
            \item Dropout能有效防止过拟合,但会轻微降低收敛速度。
            \item 卷积核大小和池化类型对性能的影响相对较小,但会影响计算效率。
        \end{enumerate}

        \textbf{8.2 未来工作}\\[0.2cm]
        未来可以从以下方向展开:
        \begin{itemize}
            \item 扩展消融研究到更多架构(如ResNet, Vision Transformer)。
            \item 研究组件之间的交互效应,设计更精细的消融实验。
            \item 探索自动化消融研究框架,降低实验成本。
        \end{itemize}
        \vspace{0.3cm}
    \end{minipage}
}
\end{center}

% -----------------------------------------------------------------------------
% 9. 参考文献
% -----------------------------------------------------------------------------
\subsection*{9. 参考文献}

% 注释:列出引用的相关文献,格式可参照主文档的bibliography样式。

\begin{center}
\fbox{
    \begin{minipage}[t]{\textwidth}
        \vspace{0.3cm}
        \begin{thebibliography}{9}
        \bibitem{simonyan2014very}
        Karen Simonyan and Andrew Zisserman, "Very deep convolutional networks for large-scale image recognition," \textit{arXiv preprint arXiv:1409.1556}, 2014.

        \bibitem{he2016deep}
        Kaiming He, Xiangyu Zhang, Shaoqing Ren, and Jian Sun, "Deep residual learning for image recognition," in \textit{Proceedings of the IEEE Conference on Computer Vision and Pattern Recognition (CVPR)}, 2016.

        \bibitem{ioffe2015batch}
        Sergey Ioffe and Christian Szegedy, "Batch normalization: Accelerating deep network training by reducing internal covariate shift," in \textit{International Conference on Machine Learning}, 2015.
        \end{thebibliography}
        \vspace{0.3cm}
    \end{minipage}
}
\end{center}

% -----------------------------------------------------------------------------
% 10. 附录(可选)
% -----------------------------------------------------------------------------
\subsection*{10. 附录(可选)}

\begin{center}
\fbox{
    \begin{minipage}[t]{\textwidth}
        \vspace{0.3cm}
        \textbf{附录A:实验环境配置}\\[0.2cm]
        \begin{itemize}
            \item 操作系统:Ubuntu 20.04 LTS
            \item Python版本:3.8.10
            \item 深度学习框架:PyTorch 1.9.0
            \item GPU:NVIDIA RTX 3090(24GB)
            \item CUDA版本:11.1
        \end{itemize}

        \textbf{附录B:代码获取}\\[0.2cm]
        本研究的完整代码已开源,可在 \url{https://github.com/username/repo} 获取。
        \vspace{0.3cm}
    \end{minipage}
}
\end{center}

% =============================================================================
% 模板结束
% =============================================================================